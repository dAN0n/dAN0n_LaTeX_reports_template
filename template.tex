% НЕ ЗАБЫВАЙТЕ МЕНЯТЬ ИМЯ ОСНОВНОГО ФАЙЛА В ПЕРВЫХ СТРОЧКАХ ВКЛЮЧАЕМЫХ ФАЙЛОВ,
% ЭТО УПРОЩАЕТ НАВИГАЦИЮ И РАБОТУ С ПРОЕКТОМ!!!
\documentclass[oneside,final,14pt]{extarticle}				% Тип документа
\usepackage{sty/SPbSTUreports}								% Стилевой файл
\addbibresource{bibl/ref.bib}								% Файл библиографической базы

%% ЗАПОЛНЕНИЕ ТИТУЛЬНОГО ЛИСТА %%
\student{Д. А. Зобков}
% \studentt{}
\group{33501/3}
\teacher{Н. В. Богач}
\papertype{лабораторной работе}
\lessontype{Телекоммуникационные технологии}
\theme{Система верстки \TeX~и расширения \LaTeX~(Шаблон для отчётов)}
% \titleright{45}
\wordfont													% если раскомменчен, при наличии используются шрифты PSCyr (устанавливаются отдельно)

% \includeonly{}											% Компиляция части отчета

\begin{document}

	\maketitle												% Генерация титульного листа
	\toc													% Генерация оглавления

	\include*{sections/goaltask}							% Цель работы/Постановка задачи
	\include*{sections/theory}								% Теоретический раздел
	\include*{sections/lab}									% Ход работы
	\include*{sections/conclusion}							% Выводы

	\include*{sections/body}								% Случаи из практики автора шаблона
	\lib													% Генерация списка литературы
	\include*{sections/appendix}							% Приложения

\end{document}