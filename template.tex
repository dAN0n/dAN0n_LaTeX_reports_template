% НЕ ЗАБЫВАЙТЕ МЕНЯТЬ ИМЯ ОСНОВНОГО ФАЙЛА В ПЕРВЫХ СТРОЧКАХ ВКЛЮЧАЕМЫХ ФАЙЛОВ,
% ЭТО УПРОЩАЕТ НАВИГАЦИЮ И РАБОТУ С ПРОЕКТОМ!!!
\documentclass[oneside,final,14pt]{extarticle}				% Тип документа
\usepackage{sty/SPbSTUreports}								% Стилевой файл
\addbibresource{bibl/ref.bib}								% Файл библиографической базы

%% ЗАПОЛНЕНИЕ ТИТУЛЬНОГО ЛИСТА %%
\university{Санкт-Петербургский государственный политехнический \\ университет Петра Великого}
% \faculty{}
\department{Кафедра компьютерных систем и программных технологий}
\student{И. И. Иванов}
% \studentt{}
\group{33501/3}
\teacher{В. И. Пупкин}
% \teacherr{}
\papertype{лабораторной работе}
\lessontype{Программирование}
\theme{Шаблон для отчётов в системе верстки \TeX~и расширения \LaTeX}
% \variant[~\textnumero]{2}
% \titleright{45}
\wordfont													% если раскомменчен, при наличии используются шрифты PSCyr (устанавливаются отдельно)

% \includeonly{}											% Компиляция части отчета

\begin{document}

	\maketitle												% Генерация титульного листа
	\toc													% Генерация оглавления

	\include*{sections/goaltask}							% Цель работы/Постановка задачи
	\include*{sections/theory}								% Теоретический раздел
	\include*{sections/lab}									% Ход работы
	\include*{sections/conclusion}							% Выводы

	\include*{sections/body}								% Случаи из практики автора шаблона
	\lib													% Генерация списка литературы
	\include*{sections/appendix}							% Приложения

\end{document}